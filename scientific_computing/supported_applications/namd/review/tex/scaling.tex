% % % % \input{tex/scaling.tex}
\documentclass[11pt]{article}

\usepackage[a4paper,includeheadfoot,top=1cm,bottom=2cm,left=2cm,right=2cm]{geometry}
\usepackage{here}
\usepackage{sectsty}
\usepackage[colorlinks=true,urlcolor=blue]{hyperref}
\usepackage{graphicx}
\usepackage{gnuplottex}

\section{Representative Benchmarks and Scaling}
Please report in this section the mandatory strong scaling tests performed using the selected code to address the proposed study 
with a representative system to be investigated during the research work. 
The scope of this test is to choose the most efficient job size to run the performance analysis on your representative system 
that will be reported in the next section. 
You should therefore select meaningful job sizes for your representative system, compatible with reasonably short runtimes: 
the lowest number of nodes might be determined by memory and wall-time constraints, while the highest number of nodes tested 
should allow you identify the job size at which you reach $\sim 50\%$ of the parallel efficiency, given by the ideal scaling. 
Weak scaling tests might be provided as well, in addition to the strong scaling scalability data. 

Figure~\ref{fig:scaling} shows the scalability plot for our example system: we start the scalability study with a job size of 8 nodes, 
whose runtime will be used as reference to compute the speed-up of larger job sizes. 
We then proceed doubling the number of nodes and checking the corresponding speed-up, until we are sure to have reached the $50\%$ 
limit in parallel efficiency. 

\begin{figure}[H]
 \begin{center}
  \begin{gnuplot}
   % gnuplot script file to create a scalability plot
   set terminal latex
   set xlabel "Number of Nodes"
   set ylabel "Speed-up"
   set key left
   set size 1.,1.
   plot "-" w lp title "Representative system", x/8 w lines lt 3 title "Ideal Speed-up"
   8 1.0
   16 1.9
   32 3.7
   64 6.0
   128 7.5
   e
   8 1.0
   16 1.75
   32 3.0
   64 5.0
   128 7.0

% cn / speedup / ideal speedup / efficiency(%) / wallt(seconds) / cores
1   1.0 1   100     83.9    12
2   1.5 2   77      54.2    24
4   2.2 4   56      37.5    48
8   3.0 8   37      28.3    96
16  3.4 16  21      24.5    192
32  3.8 32  12      22.3    384
64  3.2 64  5       26.0    768
"daint-gpu (01/2017) NAMD/2.11-CrayIntel-2016.11-cuda-8.0.054"









  \end{gnuplot}
 \end{center}
 \caption{Scalability test: strong scaling efficiency vs. ideal speed-up for the representative system}
 \label{fig:scaling}
\end{figure}

%  \section{Performance Analysis}
%  We report in the present section a summary of the performance analysis conducted on the representative system 
%  at the job size selected in the previous section, which reached $\sim 50\%$ parallel efficiency.
%  We have followed the instructions available on the section 
%  \href{http://usertest.cscs.ch/scientific_computing/performance_report}{Performance Report} 
%  of the \href{user.cscs.ch}{CSCS User Portal}, in order to instrument the executable of the selected code 
%  with \emph{Cray Performance and Analysis Tools}. 
%  The results of the run with the instrumented executable are the report file with extension \verb!.rpt! which we attach at the time of the 
%  proposal submission and the larger apprentice file with extension \verb!.ap2!: 
%  we will save both files on our \verb!$HOME! or \verb!$PROJECT! folder and make them readable to allow CSCS technical reviewers inspect them.  
%  We have documented a single production system: in case your project is split into multiple production runs with the same resources requirements, 
%  then please choose the most representative of them. Please find below the required summary data extracted from the report file:
%  
%  \begin{verbatim}
%  Experiment:                   lite  lite/gpu     
%  Number of PEs (MPI ranks):      16
%  Numbers of PEs per Node:         1  PE on each of  16  Nodes
%  Numbers of Threads per PE:   1,114
%  Number of Cores per Socket:     12
%  Execution start time:  Tue Mar 28 15:15:55 2017
%  System name and speed:  nid02294  2601 MHz (approx)
%  Intel haswell CPU  Family:  6  Model: 63  Stepping:  2
%  
%  
%  Avg Process Time:     2,100 secs             
%  High Memory:       13,977.3 MBytes     873.6 MBytes per PE
%  I/O Read Rate:    67.110363 MBytes/sec       
%  
%  |  31.8% |   664.415775 |         -- |    -- |     35,648.0 |MPI_SYNC
%  |   2.8% |    58.511390 |         -- |    -- | 14,458,788.1 |MPI
%  
%  |  59.2% | 1,236.266484 | 110.728787 |  8.8% |          1.0 |USER
%  \end{verbatim}

\end{document}

\documentclass[11pt]{article}

\usepackage[a4paper,includeheadfoot,top=1cm,bottom=2cm,left=2cm,right=2cm]{geometry}
\usepackage{here}
\usepackage{sectsty}
\usepackage[colorlinks=true,urlcolor=blue]{hyperref}
\usepackage{graphicx}
\usepackage{gnuplottex}

\section{Representative Benchmarks and Scaling}
Please report in this section the mandatory strong scaling tests performed using the selected code to address the proposed study 
with a representative system to be investigated during the research work. 
The scope of this test is to choose the most efficient job size to run the performance analysis on your representative system 
that will be reported in the next section. 
You should therefore select meaningful job sizes for your representative system, compatible with reasonably short runtimes: 
the lowest number of nodes might be determined by memory and wall-time constraints, while the highest number of nodes tested 
should allow you identify the job size at which you reach $\sim 50\%$ of the parallel efficiency, given by the ideal scaling. 
Weak scaling tests might be provided as well, in addition to the strong scaling scalability data. 

Figure~\ref{fig:scaling} shows the scalability plot for our example system: we start the scalability study with a job size of 8 nodes, 
whose runtime will be used as reference to compute the speed-up of larger job sizes. 
We then proceed doubling the number of nodes and checking the corresponding speed-up, until we are sure to have reached the $50\%$ 
limit in parallel efficiency. 

\begin{figure}[H]
 \begin{center}
  \begin{gnuplot}
   % gnuplot script file to create a scalability plot
   set terminal latex
   set xlabel "Number of Nodes"
   set ylabel "Speed-up"
   set key left
   set size 1.,1.
   plot "-" w lp title "Representative system", x/8 w lines lt 3 title "Ideal Speed-up"
   8 1.0
   16 1.9
   32 3.7
   64 6.0
   128 7.5
   e
   8 1.0
   16 1.75
   32 3.0
   64 5.0
   128 7.0

% cn / speedup / ideal speedup / efficiency(%) / wallt(seconds) / cores
1   1.0 1   100     83.9    12
2   1.5 2   77      54.2    24
4   2.2 4   56      37.5    48
8   3.0 8   37      28.3    96
16  3.4 16  21      24.5    192
32  3.8 32  12      22.3    384
64  3.2 64  5       26.0    768
"daint-gpu (01/2017) NAMD/2.11-CrayIntel-2016.11-cuda-8.0.054"









  \end{gnuplot}
 \end{center}
 \caption{Scalability test: strong scaling efficiency vs. ideal speed-up for the representative system}
 \label{fig:scaling}
\end{figure}

%  \section{Performance Analysis}
%  We report in the present section a summary of the performance analysis conducted on the representative system 
%  at the job size selected in the previous section, which reached $\sim 50\%$ parallel efficiency.
%  We have followed the instructions available on the section 
%  \href{http://usertest.cscs.ch/scientific_computing/performance_report}{Performance Report} 
%  of the \href{user.cscs.ch}{CSCS User Portal}, in order to instrument the executable of the selected code 
%  with \emph{Cray Performance and Analysis Tools}. 
%  The results of the run with the instrumented executable are the report file with extension \verb!.rpt! which we attach at the time of the 
%  proposal submission and the larger apprentice file with extension \verb!.ap2!: 
%  we will save both files on our \verb!$HOME! or \verb!$PROJECT! folder and make them readable to allow CSCS technical reviewers inspect them.  
%  We have documented a single production system: in case your project is split into multiple production runs with the same resources requirements, 
%  then please choose the most representative of them. Please find below the required summary data extracted from the report file:
%  
%  \begin{verbatim}
%  Experiment:                   lite  lite/gpu     
%  Number of PEs (MPI ranks):      16
%  Numbers of PEs per Node:         1  PE on each of  16  Nodes
%  Numbers of Threads per PE:   1,114
%  Number of Cores per Socket:     12
%  Execution start time:  Tue Mar 28 15:15:55 2017
%  System name and speed:  nid02294  2601 MHz (approx)
%  Intel haswell CPU  Family:  6  Model: 63  Stepping:  2
%  
%  
%  Avg Process Time:     2,100 secs             
%  High Memory:       13,977.3 MBytes     873.6 MBytes per PE
%  I/O Read Rate:    67.110363 MBytes/sec       
%  
%  |  31.8% |   664.415775 |         -- |    -- |     35,648.0 |MPI_SYNC
%  |   2.8% |    58.511390 |         -- |    -- | 14,458,788.1 |MPI
%  
%  |  59.2% | 1,236.266484 | 110.728787 |  8.8% |          1.0 |USER
%  \end{verbatim}

\end{document}

\documentclass[11pt]{article}

\usepackage[a4paper,includeheadfoot,top=1cm,bottom=2cm,left=2cm,right=2cm]{geometry}
\usepackage{here}
\usepackage{sectsty}
\usepackage[colorlinks=true,urlcolor=blue]{hyperref}
\usepackage{graphicx}
\usepackage{gnuplottex}

\section{Representative Benchmarks and Scaling}
Please report in this section the mandatory strong scaling tests performed using the selected code to address the proposed study 
with a representative system to be investigated during the research work. 
The scope of this test is to choose the most efficient job size to run the performance analysis on your representative system 
that will be reported in the next section. 
You should therefore select meaningful job sizes for your representative system, compatible with reasonably short runtimes: 
the lowest number of nodes might be determined by memory and wall-time constraints, while the highest number of nodes tested 
should allow you identify the job size at which you reach $\sim 50\%$ of the parallel efficiency, given by the ideal scaling. 
Weak scaling tests might be provided as well, in addition to the strong scaling scalability data. 

Figure~\ref{fig:scaling} shows the scalability plot for our example system: we start the scalability study with a job size of 8 nodes, 
whose runtime will be used as reference to compute the speed-up of larger job sizes. 
We then proceed doubling the number of nodes and checking the corresponding speed-up, until we are sure to have reached the $50\%$ 
limit in parallel efficiency. 

\begin{figure}[H]
 \begin{center}
  \begin{gnuplot}
   % gnuplot script file to create a scalability plot
   set terminal latex
   set xlabel "Number of Nodes"
   set ylabel "Speed-up"
   set key left
   set size 1.,1.
   plot "-" w lp title "Representative system", x/8 w lines lt 3 title "Ideal Speed-up"
   8 1.0
   16 1.9
   32 3.7
   64 6.0
   128 7.5
   e
   8 1.0
   16 1.75
   32 3.0
   64 5.0
   128 7.0

% cn / speedup / ideal speedup / efficiency(%) / wallt(seconds) / cores
1   1.0 1   100     83.9    12
2   1.5 2   77      54.2    24
4   2.2 4   56      37.5    48
8   3.0 8   37      28.3    96
16  3.4 16  21      24.5    192
32  3.8 32  12      22.3    384
64  3.2 64  5       26.0    768
"daint-gpu (01/2017) NAMD/2.11-CrayIntel-2016.11-cuda-8.0.054"









  \end{gnuplot}
 \end{center}
 \caption{Scalability test: strong scaling efficiency vs. ideal speed-up for the representative system}
 \label{fig:scaling}
\end{figure}

%  \section{Performance Analysis}
%  We report in the present section a summary of the performance analysis conducted on the representative system 
%  at the job size selected in the previous section, which reached $\sim 50\%$ parallel efficiency.
%  We have followed the instructions available on the section 
%  \href{http://usertest.cscs.ch/scientific_computing/performance_report}{Performance Report} 
%  of the \href{user.cscs.ch}{CSCS User Portal}, in order to instrument the executable of the selected code 
%  with \emph{Cray Performance and Analysis Tools}. 
%  The results of the run with the instrumented executable are the report file with extension \verb!.rpt! which we attach at the time of the 
%  proposal submission and the larger apprentice file with extension \verb!.ap2!: 
%  we will save both files on our \verb!$HOME! or \verb!$PROJECT! folder and make them readable to allow CSCS technical reviewers inspect them.  
%  We have documented a single production system: in case your project is split into multiple production runs with the same resources requirements, 
%  then please choose the most representative of them. Please find below the required summary data extracted from the report file:
%  
%  \begin{verbatim}
%  Experiment:                   lite  lite/gpu     
%  Number of PEs (MPI ranks):      16
%  Numbers of PEs per Node:         1  PE on each of  16  Nodes
%  Numbers of Threads per PE:   1,114
%  Number of Cores per Socket:     12
%  Execution start time:  Tue Mar 28 15:15:55 2017
%  System name and speed:  nid02294  2601 MHz (approx)
%  Intel haswell CPU  Family:  6  Model: 63  Stepping:  2
%  
%  
%  Avg Process Time:     2,100 secs             
%  High Memory:       13,977.3 MBytes     873.6 MBytes per PE
%  I/O Read Rate:    67.110363 MBytes/sec       
%  
%  |  31.8% |   664.415775 |         -- |    -- |     35,648.0 |MPI_SYNC
%  |   2.8% |    58.511390 |         -- |    -- | 14,458,788.1 |MPI
%  
%  |  59.2% | 1,236.266484 | 110.728787 |  8.8% |          1.0 |USER
%  \end{verbatim}

\end{document}

\documentclass[11pt]{article}

\usepackage[a4paper,includeheadfoot,top=1cm,bottom=2cm,left=2cm,right=2cm]{geometry}
\usepackage{here}
\usepackage{sectsty}
\usepackage[colorlinks=true,urlcolor=blue]{hyperref}
\usepackage{graphicx}
\usepackage{gnuplottex}

\section{Representative Benchmarks and Scaling}
Please report in this section the mandatory strong scaling tests performed using the selected code to address the proposed study 
with a representative system to be investigated during the research work. 
The scope of this test is to choose the most efficient job size to run the performance analysis on your representative system 
that will be reported in the next section. 
You should therefore select meaningful job sizes for your representative system, compatible with reasonably short runtimes: 
the lowest number of nodes might be determined by memory and wall-time constraints, while the highest number of nodes tested 
should allow you identify the job size at which you reach $\sim 50\%$ of the parallel efficiency, given by the ideal scaling. 
Weak scaling tests might be provided as well, in addition to the strong scaling scalability data. 

Figure~\ref{fig:scaling} shows the scalability plot for our example system: we start the scalability study with a job size of 8 nodes, 
whose runtime will be used as reference to compute the speed-up of larger job sizes. 
We then proceed doubling the number of nodes and checking the corresponding speed-up, until we are sure to have reached the $50\%$ 
limit in parallel efficiency. 

\begin{figure}[H]
 \begin{center}
  \begin{gnuplot}
   % gnuplot script file to create a scalability plot
   set terminal latex
   set xlabel "Number of Nodes"
   set ylabel "Speed-up"
   set key left
   set size 1.,1.
   plot "-" w lp title "Representative system", x/8 w lines lt 3 title "Ideal Speed-up"
   8 1.0
   16 1.9
   32 3.7
   64 6.0
   128 7.5
   e
   8 1.0
   16 1.75
   32 3.0
   64 5.0
   128 7.0

% cn / speedup / ideal speedup / efficiency(%) / wallt(seconds) / cores
1   1.0 1   100     83.9    12
2   1.5 2   77      54.2    24
4   2.2 4   56      37.5    48
8   3.0 8   37      28.3    96
16  3.4 16  21      24.5    192
32  3.8 32  12      22.3    384
64  3.2 64  5       26.0    768
"daint-gpu (01/2017) NAMD/2.11-CrayIntel-2016.11-cuda-8.0.054"









  \end{gnuplot}
 \end{center}
 \caption{Scalability test: strong scaling efficiency vs. ideal speed-up for the representative system}
 \label{fig:scaling}
\end{figure}

%  \section{Performance Analysis}
%  We report in the present section a summary of the performance analysis conducted on the representative system 
%  at the job size selected in the previous section, which reached $\sim 50\%$ parallel efficiency.
%  We have followed the instructions available on the section 
%  \href{http://usertest.cscs.ch/scientific_computing/performance_report}{Performance Report} 
%  of the \href{user.cscs.ch}{CSCS User Portal}, in order to instrument the executable of the selected code 
%  with \emph{Cray Performance and Analysis Tools}. 
%  The results of the run with the instrumented executable are the report file with extension \verb!.rpt! which we attach at the time of the 
%  proposal submission and the larger apprentice file with extension \verb!.ap2!: 
%  we will save both files on our \verb!$HOME! or \verb!$PROJECT! folder and make them readable to allow CSCS technical reviewers inspect them.  
%  We have documented a single production system: in case your project is split into multiple production runs with the same resources requirements, 
%  then please choose the most representative of them. Please find below the required summary data extracted from the report file:
%  
%  \begin{verbatim}
%  Experiment:                   lite  lite/gpu     
%  Number of PEs (MPI ranks):      16
%  Numbers of PEs per Node:         1  PE on each of  16  Nodes
%  Numbers of Threads per PE:   1,114
%  Number of Cores per Socket:     12
%  Execution start time:  Tue Mar 28 15:15:55 2017
%  System name and speed:  nid02294  2601 MHz (approx)
%  Intel haswell CPU  Family:  6  Model: 63  Stepping:  2
%  
%  
%  Avg Process Time:     2,100 secs             
%  High Memory:       13,977.3 MBytes     873.6 MBytes per PE
%  I/O Read Rate:    67.110363 MBytes/sec       
%  
%  |  31.8% |   664.415775 |         -- |    -- |     35,648.0 |MPI_SYNC
%  |   2.8% |    58.511390 |         -- |    -- | 14,458,788.1 |MPI
%  
%  |  59.2% | 1,236.266484 | 110.728787 |  8.8% |          1.0 |USER
%  \end{verbatim}

\end{document}
